%%%%%%%%%%%%%%%%%%%%%%%%%%%%%%%%%%%%%%%%%
% Twenty Seconds Resume/CV
% LaTeX Template
% Version 1.0 (14/7/16)
%
% Original author:
% Carmine Spagnuolo (cspagnuolo@unisa.it) with major modifications by 
% Vel (vel@LaTeXTemplates.com) and Harsh (harsh.gadgil@gmail.com)
%
% License:
% The MIT License (see included LICENSE file)
%
%%%%%%%%%%%%%%%%%%%%%%%%%%%%%%%%%%%%%%%%%

%----------------------------------------------------------------------------------------
%	PACKAGES AND OTHER DOCUMENT CONFIGURATIONS
%----------------------------------------------------------------------------------------

\documentclass[letterpaper]{twentysecondcv} % a4paper for A4
\usepackage{hyperref}
\hypersetup{
	colorlinks=true,
	linkcolor=blue,
	filecolor=magenta,      
	urlcolor=cyan,
}

\urlstyle{same}

% Command for printing skill overview bubbles
\newcommand\skills{ 
~
	\smartdiagram[bubble diagram]{
        \textbf{Senior}\\\textbf{Software Engineer},
        \textbf{Embedded Linux}\\\textbf{Programming},
        \textbf{OOP}\\\textbf{Design Pattern},
        \textbf{Software}\\\textbf{Design},
        \textbf{Linux}\\\textbf{Integration},
        \textbf{FreeRTOS}\\\textbf{~~MCU~~},
        \textbf{Debugging}\\\textbf{Static Analysis}
    }
}

% Programming skill bars
\programming{{C $\textbullet$ C++ / 4}, {Java $\textbullet$ \large \LaTeX / 3.5}, {bash $\textbullet$  Python / 2}}

% Projects text
\OS{{Fedora $\textbullet$ Ubuntu $\textbullet$ CentOS / 4}}
\IDE{{Eclipse $\textbullet$ QtCreator $\textbullet$ / 4}}
\Debugging{{gdb debugging / 4}}
%        \textbf{ZCam} - A streaming module will be integrated into the vacuum cleaner robot to enable the surveillance \\
%        \textbf{Set-top box software} - Maintain and develop the set-top box software as embedded Linux systems

%----------------------------------------------------------------------------------------
%	 PERSONAL INFORMATION
%----------------------------------------------------------------------------------------
% If you don't need one or more of the below, just remove the content leaving the command, e.g. \cvnumberphone{}

\cvname{Tan. Le Minh} % Your name
\cvjobtitle{ Senior Software Engineer } % Job
% title/career

\cvlinkedin{in/tan-le-1b41b567}
\cvgithub{lmtan91}
\cvnumberphone{(+84) 908 245458} % Phone number
\cvsite{lmtan91.github.io} % Personal website
\cvmail{lmtan91@gmail.com} % Email address

%----------------------------------------------------------------------------------------

\begin{document}

\makeprofile % Print the sidebar

%----------------------------------------------------------------------------------------
%	 EDUCATION
%----------------------------------------------------------------------------------------
\section{Education}

\begin{twenty} % Environment for a list with descriptions
	\twentyitem
    	{2009 - 2013}
		{}
        {B.A.Sc., Electronics and Telecommunication \textnormal{(GPA: 7.9)}}
        {\href{http://www.hcmus.edu.vn}{University of Science, Vietnam National University}}
        {}
        %{}
        {\begin{itemize}
        \item Computer Structure, Microcontroller and Lab
        \item Data structure and Algorithms, Computer Interfacing and DAQ
        \item OOP, Mobile Device Application Programming
        \item Embedded Systems, Matlab and DSP Lab
        \end{itemize}}
	%\twentyitem{<dates>}{<title>}{<organization>}{<location>}{<description>}
\end{twenty}

%----------------------------------------------------------------------------------------
%	 EXPERIENCE
%----------------------------------------------------------------------------------------

\section{Experience}

\begin{twenty} % Environment for a list with descriptions
	\twentyitem
		{Nov 2017 -}
		{present}
		{Senior Software Engineer - Product Development}
		{VP9VN - Hanoi}
		{}
		{\begin{itemize}
		\item Project that developed the Smart Traffic Mornitoring Camera(STMC) based on \textbf{Amlogic S905X}.
		\item \textbf{Buildroot} integration for the STMC.
		\item Implement OTA using \textbf{swupdate} with double copy strategy and using \textbf{hawkbit} as the updater server.
		\item Investigate \textbf{nginx} load balancing and try implementing the static route using Lua plugin.
		\item Implement \textbf{RTSP} client using C/C++ to forward \textbf{RTSP} streamning from a camera to the cloud.
		\item Implement an app in \textbf{bootloader(u-boot)} to read environment variables from user space.
		\end{itemize}}

	\\
	\twentyitem
		{Nov 2015 -}
		{Oct 2017}
		{Senior Software Engineer - Product Development}
	    {Robert Bosch Engineering and Business Solutions}
	    {}
	    {\begin{itemize}
	    \item Project that developed the streaming system that mounted on the vacuum cleaning robot as
	    surveillance with motion detection.\url{https://www.youtube.com/watch?v=VIsczH4iZwM}
	    \item Study customer’s requirement and implement new specification then do the POC in evaluation kit.
	    \item Research how to bring up \textbf{i.MX6} custom board running Linux and first time bring up the
	    Linux board successfully.
	    \item Working closely with hardware team while bringup and testing hardware.
	    \item Able to understand schematic and using Oscilloscope to measure the signal.
	    \item System integration on \textbf{Yocto} and customize for the specific project.
	    \item Customize the rootfs, device tree, Linux kernel and \textbf{bootloader(u-boot)}.
	    \item Design and implement the \textbf{BIST(Built-in Self Test)} component that executes the hardware
	    diagnostic when bootup.
	    \item Do the \textbf{FOSS} (Free and Open Source Software) license checking for the product go to the
	    market.
	    \item Responsible for designing and implementing the streaming application based on \textbf{Gstreamer}
	    platform and working closely with \textbf{Wowza} streaming team to integrate with Wowza service.
	    \item Responsible for setting up the \textbf{PC-Lint} (Commercial Static Analysis Tool) on the GNU project and fix the error/warning/info reported by PC-Lint.
	    \end{itemize}}
        \end{twenty}
        \newpage
        \makeprofile
                \section{Experience}
        \begin{twenty}
	\twentyitem
    	{Sep 2016 -}
		{Nov 2016}
        {Senior Software Engineer - COC HMI}
        {{Robert Bosch Engineering and Business Solutions}}
        {}
        {\begin{itemize}
        \item Project that supported the German Bosch to integrate the Bosch protocol into the i.MX6
        platform.
        \item Worked as the system integration engineer based on \textbf{Yocto} project.
        \item Integrated the packages support the web development such as: AngularJS, html5,...
        \item Designed and implemented the C++ application(\textbf{OOP}) as the json-rpc websocket server based
        on the \href{https://github.com/cinemast/libjson-rpc-cpp}{libjson-rpc-cpp} that will receive the json data from the front end (Ex: increase the oval cooking temperature to 40◦ )
        \item Applied \textbf{design pattern} in software design.
        \end{itemize}}
    \end{twenty}

        \begin{twenty}
	\twentyitem
    	{Sep 2015 -}
		{Nov 2015}
        {Embedded Software Engineer}
        {{Robert Bosch Engineering and Business Solutions}}
        {}
        {
        {\begin{itemize}
        \item Worked \textbf{onsite in India} to support the lazer distance measuring project.
        \item Setup the eclipse debugging for the project and modified the Makefile to support gdb
        debugging.
        \item Researched open source static analyzer tools: cppcheck, clang,... and apply to the project.
        \item Report the \textbf{FOSS} checking in Yocto for the project.
        \item Measure the memory footprint in the system using Python.
	    \end{itemize}}
        }
\\
	\twentyitem
	{Sep 2013 -}
	{Sep 2015}
	{Embedded Software Engineer}
	{\href{http://www.globalcybersoft.com/en}{Global Cybersoft Vietnam}}
	{}
	{
		{\begin{itemize}
				\item Maintained and developed the set-top box software for \textbf{DIRECTV(AT\&T)}.
				\item Studied and researched the embedded Linux architecture.
				\item Had the chance to study really the perfect embedded Linux product.
				\item Understood actual design concepts: design for test, design for debugging.
				\item Understood and design the software architecture from component design, class diagram, sequence diagram.
		\end{itemize}}
	}
	\\
	\twentyitem
	{Jun 2012 -}
	{Aug 2012}
	{Internship}
	{{Renesas Design Vietnam}}
	{}
	{
		\begin{itemize}
			\item One of 16 students won the technical exam and interviews of the company to get the internship position at the company.
			\item Studied the 8-bit MCU designed by Renesas, compiler, IDE and hard debugging.
			\item Was responsible for all lab testing routines involving all peripherals in the specific MCU.
			\item Designed and implemented the digital watch project with Lunar calendar feature.
			\item Besides that, improved the soft skills such as: presentation, team work, English.
		\end{itemize}
	}
\end{twenty}
	    \\   
	    \newpage
\section{Hobby Projects}
\begin{twenty}
	\twentyitem
	{2016}
	{}
	{Porting Clutter project on Beaglebone Black}
	{\href{https://blogs.gnome.org/clutter/}{Clutter Project}}
	{}
	{
		{\begin{itemize}
		\item Repository: \url{https://github.com/lmtan91/meta-lmtan91/tree/master/clutter}
		\item Using Yocto to port Clutter into Beaglebone Black in case of creating nice GUI.
		\end{itemize}}
	}
	\\
	\twentyitem
	{2016}
	{}
	{Porting LWIP and FreeRTOS into STM32F4 Discovery kit}
	{\href{https://github.com/lmtan91/STM32\_ETH\_LWIP}{STM32\_ETH\_LWIP}}
	{}
	{
		{\begin{itemize}
		\item Repository: \url{https://github.com/lmtan91/STM32\_ETH\_LWIP}
		\item Porting to use GNU GCC compiler, OpenOCD for debugging, Eclipse as IDE for free without using Commercial Compiler and IDE.
		\item Research FreeRTOS and LWIP by implementing some applications as web server.
		\end{itemize}}
	}
%\\
\end{twenty}

\section{Scholarship}
\begin{twenty}
	\twentyitem
	{2012}
	{}
	{Sunflower Mission 2012 Engineering \& Technology Scholarship for Excellence}
	{\href{https://www.esilicon.com/company/careers/scholarships-in-vietnam}{Sunflower Mission}}
	{}
	
\end{twenty}
\end{document} 
